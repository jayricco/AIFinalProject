\documentclass[10pt,letterpaper]{article}
\usepackage[latin1]{inputenc}
\usepackage{amsmath}
\usepackage{amsfonts}
\usepackage{amssymb}
\usepackage{makeidx}
\usepackage{graphicx}
\usepackage{algorithm2e}
\usepackage[left=1.00in, right=1.00in, top=1.00in, bottom=1.00in]{geometry}
\author{Jay Ricco, David Van Chu}
\title{Introduction to Artificial Intelligence\\MNIST and CIFAR10 with Boundary Trees}
\begin{document}
	\maketitle
	\section{MNIST}
		Changing the maximum branching factor, k, played a large role in both accuracy as well as the speed of training and querying. As k grows, it takes longer to train and query, although the accuracy is higher, so it may be worth it depending on the use case.
		For example:
		\begin{enumerate}
			\item k = $\infty$
			\item accuracy = 88.2$\%$
			\item training = 156.22 seconds
			\item testing = 33.97 seconds
		\end{enumerate}
	
		\begin{enumerate}
			\item k = $\infty$
			\item accuracy = 85$\%$
			\item training = 72.21 seconds
			\item testing = 14.34 seconds
		\end{enumerate}
	\section{CIFAR10}
		
\end{document}